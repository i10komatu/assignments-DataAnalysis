\documentclass{jsarticle}

\usepackage[top=30mm, bottom=36mm, left=28mm, right=28mm]{geometry}
\usepackage[yyyymmdd]{datetime}
\usepackage{listings}
\usepackage{../common/mytitle}

\title{データ解析特論 第1回}
\author{201720690 小松 弘人}
\date{\today}
\pagestyle{empty}

\makeatletter
\def\mojiparline#1{
    \newcounter{mpl}
    \setcounter{mpl}{#1}
    \@tempdima=\linewidth
    \advance\@tempdima by-\value{mpl}zw
    \addtocounter{mpl}{-1}
    \divide\@tempdima by \value{mpl}
    \advance\kanjiskip by\@tempdima
    \advance\parindent by\@tempdima
}
\makeatother
\def\linesparpage#1{
    \baselineskip=\textheight
    \divide\baselineskip by #1
}

\begin{document}
\maketitle
\thispagestyle{empty}
\subsection*{問題}
R-tips P.53の\verb|matpow|関数を用いて計算した正方行列$A$の$n$乗の値が,
行列の$n$乗を\verb|A%*%A%*%…%*%A|と計算した結果と一致するかを確認せよ.
一致しない場合には,その原因がどこにあるか考察せよ.

\subsection*{matpow関数}
\begin{lstlisting}[basicstyle=\ttfamily\footnotesize, frame=single]
> matpow <- function(x, pow=2) {
+   y <- eigen(x)
+   y$vectors %*% diag( (y$values)^pow ) %*% t(y$vectors)
+ }
\end{lstlisting}

\subsection*{実行結果}
\subsubsection*{対角行列の場合}
\begin{lstlisting}[basicstyle=\ttfamily\footnotesize, frame=single]
> A <- diag(1:2)
> matpow(A, 3)
     [,1] [,2]
[1,]    1    0
[2,]    0    8
> A %*% A %*% A
     [,1] [,2]
[1,]    1    0
[2,]    0    8
\end{lstlisting}
R-tipsに掲載されている通り、$A$が対角行列の場合、結果は一致する。

\subsubsection*{対角行列でない場合}
\begin{lstlisting}[basicstyle=\ttfamily\footnotesize, frame=single]
> A <- matrix(c(8,4,1,5), 2,2)
> matpow(A, 3)
         [,1]     [,2]
[1,] 368.2647 349.4412
[2,] 349.4412 424.7353
> A %*% A %*% A
     [,1] [,2]
[1,]  596  133
[2,]  532  197
\end{lstlisting}

結果は一致しなかった。
これは、$A$の固有値分解の式が正しくないからである。
R-tipsでは、$A$の固有値分解の式は$A = VD{}^t\!V$とされている。
これが正しければ、\verb|matpow(A, 1)|は$A$と一致するはずである。
しかし、以下のように\verb|matpow(A, 1)|は$A$と一致しない。
したがって、固有値分解の式が原因であると考えられる。
\begin{lstlisting}[basicstyle=\ttfamily\footnotesize, frame=single]
> A <- matrix(c(8,4,1,5), 2,2)
> matpow(A, 1)
         [,1]     [,2]
[1,] 4.735294 3.558824
[2,] 3.558824 8.264706
\end{lstlisting}

\subsection*{修正matpow関数}
\subsubsection*{修正内容}
正しい固有値分解の式は、$A = VDV^{-1}$である。
したがって、\verb|matpow|関数を以下のように修正することで、
結果が一致するようになる。
\begin{lstlisting}[basicstyle=\ttfamily\footnotesize, frame=single]
> matpow = function(x, pow=2) {
+   y=eigen(x)
+   y$vectors %*% diag( (y$values)^pow ) %*% solve(y$vectors)
+ }
\end{lstlisting}

\subsubsection*{実行結果}
\begin{lstlisting}[basicstyle=\ttfamily\footnotesize, frame=single]
> matpow(A,1)
     [,1] [,2]
[1,]    8    1
[2,]    4    5
> A
     [,1] [,2]
[1,]    8    1
[2,]    4    5
>
> matpow(A, 3)
     [,1] [,2]
[1,]  596  133
[2,]  532  197
> A %*% A %*% A
     [,1] [,2]
[1,]  596  133
[2,]  532  197
\end{lstlisting}

\end{document}
